\ifx\empty
~
\fi

%% start of file `jdoe_casual.tex'.
%% Copyright 2006 Xavier Danaux.
%
% This work may be distributed and/or modified under the
% conditions of the LaTeX Project Public License version 1.3c,
% available at http://www.latex-project.org/lppl/.


\documentclass[letterpaper,11pt]{moderncv}
%\documentclass[a4paper,11pt]{moderncv}
\usepackage{xspace}

\usepackage{relsize}

% moderncv styles
%\moderncvstyle{casual}         % optional argument are 'nocolor' (black & white cv) and 'roman' (for roman fonts, instead of sans serif fonts)
\moderncvstyle{classic}       % idem

% character encoding
\usepackage[utf8]{inputenc}   % replace by the encoding you are using

% personal data (the given example is exhaustive; just give what you want)
\firstname{Dr.-ing.\ Ingo}
\familyname{Wald}
\title{
Tech Lead Software Defined Visualization
\newline
Intel Technical Computing Group\vspace*{2mm}
\newline
Adjunct Assistant Professor
\newline
School of Computing, University of Utah
\vspace*{2mm}
}

% \title{Research Assistant Professor 
% \vspace{2mm}
% \newline
% School of Computing
% \vspace{1mm}
% \newline
% University of Utah}
% \title{
% Research Scientist, 
% \newline
% \hspace*{4mm}  
% Advanced Graphics Architectures, Intel Corp.
% \vspace{2mm}
% \newline
% %Adjunct Professor, 
% Research Assistant Professor, 
% \newline
% \hspace*{4mm} School of Computing, University of Utah}

%\address{12 somestreet\\3456 somecity} % for classic style
\addressone{Ingo Wald}
\addresstwo{10631 Chestnut Ridge Rd}  % for casual style
\addressthree{Austin, TX, 78726}  % for casual style
%\addresstwo{2200 Mission College Blvd \#302}  % for casual style
%\addressthree{Santa Clara, CA, 95054}  % for casual style
%\phone{+49 172 841 9971}
\phone{(801) 556 2503}
\email{\url{Ingo.Wald@intel.com}}
\extrainfo{\url{www.sci.utah.edu/~wald}}
%\photo[85pt]{wald}
%\photo[64pt]{wald_sci}

%\renewcommand{\listsymbol}{{\fontencoding{U}\fontfamily{ding}\selectfont\tiny\symbol{'102}}} % define another symbol to be used in front of the list items

% the ConTeXt symbol
\def\ConTeXt{%
  C%
  \kern-.0333emo%
  \kern-.0333emn%
  \kern-.0667em\TeX%
  \kern-.0333emt}

% slanted small caps (only with roman family; the sans serif font doesn't exists :-()
%\usepackage{slantsc}
%\DeclareFontFamily{T1}{myfont}{}
%\DeclareFontShape{T1}{myfont}{m}{scsl}{ <-> cork-lmssqbo8}{}
%\usefont{T1}{myfont}{m}{scsl}Testing the font

% command and color used in this document, independently from moderncv 
\definecolor{see}{rgb}{0.5,0.5,0.5}% for web links
\newcommand{\up}[1]{\ensuremath{^\textsf{\scriptsize#1}}}% for text subscripts

%----------------------------------------------------------------------------------
%            content
%----------------------------------------------------------------------------------
\begin{document}
\vspace*{-1em}
\maketitle
\vspace*{-3em}
\makequote


\def\ende#1#2{#1}
%\def\ende#1#2{#2}
\def\in{In\xspace}
\def\proc{\in Proceedings of\xspace}

\def\mit{\ende{with}{mit}\xspace}
\def\with{\ende{with}{mit}\xspace}
\def\und{\ende{and}{und}\xspace}
\def\and{\ende{and}{und}\xspace}
\def\seit{\ende{since}{seit}\xspace}
\def\pages{\ende{pages}{S.}\xspace}
\def\indeutsch{\ende{(in german)}{(in Deutsch)}\xspace}
\def\may{\ende{May}{Mai}\xspace}
\def\mar{\ende{Mar}{M\"arz}\xspace}

\def\meanwhileavailable{\emph{Submitted for publication,} meanwhile available as\xspace}
\def\submitted{Submitted for publication}

\def\cvbibitem#1#2#3#4{\cvitem{{$\color{titlecolor}\triangleright$}}{{\bfseries
      #1.\ }{\ifthenelse{\equal{#2}{}}{}{#2.\ }}{\ifthenelse{\equal{#3}{}}{}{\emph{#3}.\ }}{\ifthenelse{\equal{#4}{}}{}{#4.\xspace}}}}


\def\cvownitem#1#2#3{\cvitem{{$\color{titlecolor}\triangleright$}}{{\bfseries
      #1.\ }{\ifthenelse{\equal{#2}{}}{}{\em #2.\ }}{\ifthenelse{\equal{#3}{}}{}{#3.\xspace}}}}

\vspace*{-4mm}

%\firstnamestyle{\Huge{Curriculum Vitae Dr.-ing.\ Ingo Wald}}
%\vspace*{4mm}
%\begin{center}
%{\color{sectiontitlecolor} \relsize{+4}{{Curriculum Vitae Dr.-ing.\ Ingo Wald}}}
%\end{center}

\vspace{-12mm}
\section{Personal Data}
\cvitem{Name}{Ingo Wald}
\cvitem{Born}{March 29, 1974, in Bad Kreuznach, Germany\\Nationality: German; Residency: US (Permant Resident Card)}
%\cvitem{Marital Status}{Married, three children.}

\vspace*{-3mm}
\section{Scientific Achievements in Brief}
\vspace*{-1mm}
\cvitem{Publications}{4 patents, 23 journal articles; 40~conference
  papers; 3~peer-reviewed STAR reports; 2~conference proceedings (edited); 
%3~SIGGRAPH courses\\
(Intel-internal papers intentionally excluded)}
\cvitem{Academic Impact}{Citation count (scholar.google.com) for most cited
  paper (Oct): $>500$.
\\
h-index (\#publications with more than N citations): 37 (google scholar, Oct'15).}


\vspace*{-3mm}

\section{Professional Experience}
\vspace*{-1mm}
\cventry{2013--current}
{Intel Technical Computing Group (Data Center Group), Intel Corp}
{Senior Research Scientist}
{Tech Lead 
High-Fidelity and 
Software Defined Visualization
}
{}
{}
\cventry{2007--2013}
{Intel Labs, Intel Corp}
{Senior Research Scientist}
{}
{}
{}
\cventry{2005--2007}
{Scientific Computing and Imaging
(SCI) Institute and School of Computing, University of Utah}
{Research Assistant Professor (adjunct appointment still active}
{}
{}
{}
\cventry{2004--2005}
{Max Planck Institut Informatik Saarbr\"ucken}
{Research Associate (Post-Doc)}
{}
{}
{}
\cventry{2000--2004}
{Saarland University}
{Research Associate%
% (wissenschaftlicher Mitarbeiter)
, Computer Graphics Group}
{}
{}
{
}
\closesection


\vspace*{-4mm}

\section{Education}
\vspace*{-1mm}
\cventry{2000--2004}{
Dr.-ing. (PhD equivalent)%
%PhD in Computer Science
}{Saarland University,
  Saarbr\"ucken, Germany}
{}{}{Awarded Degree: Dr.-ing.\ mit Auszeichnung (PhD in Computer Science, with Distinction).}
\cventry{1993--1999}{Dipl-inform.(MS equivalent)}{Kaiserslautern University,
  Kaiserslautern, Germany}
{}{}{Awarded Degree: Dipl-inform, sehr gut (Masters in Computer
  Science, A equivalent).}
\cventry{1984-1993}{Abitur (High-School equivalent)}{Alfred Delp Schule Hargesheim, Germany}
{}{}{Awarded Degree: Allgemeine Hochschulreife (Abitur), sehr gut
  (High School, A equivalent).}



\vspace*{-2mm}
\section{Research Interests}
\vspace*{-1mm}
\cvlistitem{
Efficient Computing on 
%General Purpose Graphics
High-Throughput Compute Architectures (MIC/Xeon Phi)}
%
%\cvlistitem{Innovate compute platforms that combine many heterogeneous
%  nodes, wide-SIMD/SIMT, many-core, \dots}
%
%
\cvlistitem{Ray Tracing 
(in particular, using Embree and OSPRay)}
\cvlistitem{Parallel and High-Performance Rendering (in part.\ using ray tracing)}
\cvlistitem{Ray Tracing for High-Fidelity Visualization}
%large-scale and time-varying data}
\cvlistitem{Photo-realistic Rendering / Monte-Carlo lighting simulation}
%\cvlistitem{Parallel and distributed rendering}
\cvlistitem{SPMD Compilers and Languages for High-Performance Computing Architectures}
\cvlistitem{Efficient
% hierarchical
 data structures and
% traversal
  algorithms, in particular on innovative HPC platforms
}
% (SIMT, wide-SIMD, many-core, clusters, heterogeneous, \dots)}
%\cvlistdoubleitem{Double\dots{}}{\dots{} item.}
%\cvlistdoubleitem{Another double\dots{}}{\dots{} item.}


%\vspace{2em}
%\flushright
%\hspace{-30mm}
\closesection{}
Austin, TX, Oct 10, 2015
\vspace*{-2em}

\closesection{}
%\pagebreak
\cleardoublepage



% =======================================================
% =======================================================
% =======================================================
\firstnamestyle{\Huge{Publications}}
\vspace*{6mm}

% =======================================================
\section{Patents}

\cvbibitem
{Ray Tracing a Three Dimensional Scene using a Grid}
{Ingo Wald, Thiago Ize, Steven G Parker, and Aaron Knoll}{US 8384711 B2}{Filed 11/06, Issued 2/2013}

\cvbibitem
{Ray Tracing A Three-Dimensional Scene using A Hierarchical Data Structure}
{Ingo Wald, Peter Shirley \und Solomon Boulos}
{US 8259105 B2 / US 20100060634 A1}
{Filed 7/2007, Issued 9/2012}

\cvbibitem
{Parallel Grid Population}
{Ingo Wald and Thiago Ize}
{WO 2008067490 A2}
{Filed 11/2007, Issued 6/2008}

\cvbibitem
{Method and device for creating a two-dimensional representation of a three dimensional world}
{J\"org Schmittler, Ingo Wald, \und Philipp Slusallek}
{US Patent Number 7289118}{Universtit\"at des Saarlandes \und Garching Innovation Technologien
      aus der Max-Planck Gesellschaft. Filed Aug 20, 2003; Issued Oct 30, 2007}{}


\closesection
\medskip


% =======================================================
\section{Books \& Proceedings (edited)}

\cvbibitem
{High-Performance Graphics 2014}{Ingo Wald and Jonathan Ragan-Kelley, editors}{}{}

\cvbibitem
{Proceedings of the 2006 IEEE Symposium on Interactive Ray
  Tracing}{Ingo Wald and Steven G.\ Parker, editors}{}{IEEE Computer Society Press, ISBN 1-4244-0693-5, 2006}


\closesection
%\medskip
\vspace{-1ex}


% =======================================================
\section{Book Chapters}

\cvbibitem
{Chapter 21 -- High Performance Ray Tracing}{Gregory S. Johnson, Ingo Wald, Sven Woop, Carsten Benthin, and Manfred Ernst}{in \emph{High Performance Parallelism Pearls: Multicore and Many-core Programming Approaches}, James Reinders and Jim Jeffers (editors), Morgan Kaufman}{2015}

\closesection

\vspace{-1ex}
% =======================================================
\section{Company-Internal Papers and Whitepapers}

% -------------------------------------------------------
  \cvbibitem 
    {(confidential, intentionally omitted)}
    {}
    {}{}

\vspace{-1ex}
% =======================================================
\section{Journal Articles (peer reviewed)}


% -------------------------------------------------------
  \cvbibitem {An Evaluation of Multi-Hit Ray Traversal in a BVH using Existing First-Hit/Any-Hit Kernels}
{Jefferson Amstutz, Christiaan Gribble, Johannes Guenther, and Ingo Wald}
{Journal of Computer Graphics Techniques}{2015 (to appear)}
  
% -------------------------------------------------------
  \cvbibitem 
    {Embree--A 
 Kernel Framework for Efficient CPU Ray Tracing}
    {Ingo Wald, Sven Woop, Carsten Benthin, Gregory S. Johnson, and Manfred Ernst}
    {ACM Transactions on Graphics (Proceedings of ACM SIGGRAPH)}{2014}

% -------------------------------------------------------
  \cvbibitem 
    {RBF Volume Ray Casting on Multicore and Manycore CPUs}
    {Aaron Knoll, Ingo Wald, Paul Navratil, Anne Bowen, Khairi Reda, Michael E Papka, and Kelly P Gaither}
    {Computer Graphics Forum (Proceedings of EuroVis)}{2014 (to appear)}

% -------------------------------------------------------
  \cvbibitem 
    {Water-tight Ray-Triangle Intersection}
    {Sven Woop, Carsten Benthin, Ingo Wald}
    {Journal of Computer Graphics Techniques}{2013}

% -------------------------------------------------------
  \cvbibitem 
    {Combining Single and Packet Ray Tracing for Arbitrary Ray Distributions on the Intel\textsuperscript\textcopyright MIC Architecture}
    {Carsten Benthin, Ingo Wald, Sven Woop, Manfred Ernst, William R. Mark}
    {IEEE Transactions on Visualization and Computer Graphics}{2012}

% -------------------------------------------------------
  \cvbibitem 
    {Fast Construction of SAH BVHs on the Intel\textsuperscript\textcopyright Many Integrated Core (MIC) Architecture}
    {Ingo Wald}
    {IEEE Transactions on Visualization and Computer Graphics}{2011}


% -------------------------------------------------------
  \cvbibitem 
    {State of the Art in Ray Tracing Animated Scenes}{Ingo Wald,
    William R Mark, Johannes G\"unther, Solomon Boulos, Thiago Ize,
    Warren Hunt, Steven G.\ Parker, and Peter Shirley}{Computer Graphics Forum}{Volume 28, Number 6, 2009}



% -------------------------------------------------------
\cvbibitem{Sequential Monte Carlo Adaption in Low-Anisotropy Participating Media} {Vincent Pegoraro, Ingo Wald, \und Steven G Parker}{Computer Graphics Forum (Proceedings of the 2008 Eurographics Symposium on Rendering)}
{2008}


% -------------------------------------------------------
\cvbibitem{Coherent Multiresolution Isosurface Ray Tracing} {Aaron
  Knoll, Charles D Hansen, \und Ingo Wald}{The Visual Computer
}{Volume 25, Number 3, March 2009}

% -------------------------------------------------------
\cvbibitem{Fast, Parallel, and Asynchronous Construction of BVHs for
  Ray Tracing Animated Scenes} { Ingo Wald, Thiago Ize, \und Steven
  G.\ Parker}{Computers and Graphics}{Volume 32, Number 1, Feb 2008}


% -------------------------------------------------------
\cvbibitem{Exploring a Boeing 777 -- Ray Tracing Large-Scale CAD Data}
          { Andreas Dietrich, Abe Stephens, \und Ingo Wald}{IEEE
            Computer Graphics and Applications}{Volume 27, Issue 6,
            (November 2007), pages 36--46}

% -------------------------------------------------------
\cvbibitem{Interactive Isosurface Ray Tracing of Time-Varying
  Tetrahedral Volumes} { Ingo Wald, Heiko Friedrich, Aaron Knoll, \and
  Charles D.\ Hansen}{IEEE Transactions on Visualization and Computer
  Graphics}{(Proceedings of IEEE Visualization/InfoVis 2007), Volume
  13, Issue 6 (November 2007), pages 1727-1734}

% -------------------------------------------------------
\cvbibitem{A Coherent Grid Traversal Approach to Visualizing
  Particle-based Simulation Data}
{Christiaan P.\ Gribble, Thiago Ize, Ingo Wald, Andrew E.\ Kensler, \und
  Steven G.\ Parker}{IEEE Transactions
on Visualization and Computer Graphics 13(4), 2007}{}%{pages~758--768} 

%(to appear, meanwhile available as SCI Institute, University of Utah,
%  Technical Report No UUSCI-2006-024)

% -------------------------------------------------------
\cvbibitem{Ray Tracing
  Deformable Scenes using Dynamic Bounding Volume
  Hierarchies}{Ingo Wald, Solomon Boulos, \und Peter Shirley}{ACM Transactions on
  Graphics 26(1), 2007}{pages~1--18}

% -------------------------------------------------------
\cvbibitem{Ray Tracing Animated Scenes using Coherent Grid
  Traversal}{Ingo Wald, Thiago Ize, Andrew E Kensler, Aaron Knoll, \und Steven G
  Parker}{ACM Transactions on
  Graphics 25(3), 2006}{pages~485-493, (Proceedings of ACM SIGGRAPH 2006)}

% -------------------------------------------------------
\cvbibitem{Realtime Ray Tracing for Advanced Visualization
in the Aerospace Industry}{Andreas Dietrich, Ingo Wald, Holger Schmidt,
Kristian Sons, and
Philipp Slusallek
}{Proceedings of the 5th Paderborner Workshop Augmented und Virtual Reality in der Produktentstehung
}{}

% -------------------------------------------------------
  \cvbibitem{Ray Tracing Animated Scenes using
    Motion Decomposition}{Johannes Günther, Heiko Friedrich, Ingo Wald, Hans-Peter
    Seidel, \und Philipp Slusallek}{Computer Graphics Forum 25(3), 
     2006}{\pages~517--525  (Proceedings EUROGRAPHICS 2006)}

% -------------------------------------------------------
\cvbibitem{Techniques for
  Interactive Ray Tracing of B\'ezier Surfaces}{Carsten Benthin, Ingo Wald, \und Philipp Slusallek}{Journal of Graphics
  Tools 11(2), 2006}{\pages~1--16}

% -------------------------------------------------------
  \cvbibitem{Faster Isosurface Ray Tracing using
      Implicit KD-Trees}{Ingo Wald, Heiko Friedrich, Gerd Marmitt, Philipp Slusallek,
    \und Hans-Peter Seidel}{IEEE Transactions on Visualization and
    Computer Graphics 11(5), 2005}{\pages~562--572}

% -------------------------------------------------------
  \cvbibitem{Balancing Considered Harmful -- Faster Photon Mapping using the
      Voxel Volume Heuristic}{ Ingo Wald, Johannes G\"unther, \and Philipp Slusallek}{Computer Graphics Forum 22(3), 2004}{
\pages~595--603,    (Proceedings of EUROGRAPHICS)}
%\ende{Aug 30--Sep 3,}{30. August -- 3.  September,} 2004, 

% -------------------------------------------------------
  \cvbibitem{A
      Scalable Approach to Interactive Global Illumination}{ Carsten Benthin, Ingo Wald, \und Philipp Slusallek}{Computer
    Graphics Forum 22(3), 2003}{
    \pages~621--630,
 (Proceedings of EUROGRAPHICS)}

% -------------------------------------------------------
\cvbibitem{Interactive Distributed Ray Tracing on Commodity PC Clusters --
  State of the Art and Practical Applications}{ Ingo Wald, Carsten Benthin, Andreas Dietrich, \und Philipp Slusallek}{Harald Kosch, Laszlo
  B\"osz\"ormenyi, \und Hermann Hellwagner, editors, \emph{Euro-Par
    2003}, volume 2790 of \emph{Lecture Notes in Computer Science}}{
  \pages~499--508, Springer, 2003}

% -------------------------------------------------------
  \cvbibitem{Interactive Rendering with Coherent Ray-Tracing}
{Ingo Wald, Carsten Benthin, Markus Wagner, \und Philipp
    Slusallek}{Computer Graphics Forum 20(3), 2001}{
    \pages~153--164, (Proceedings of EUROGRAPHICS 2001)}
%Manchester, UK, \ende{Sep 3--7,}{3.--7.~September,} 
\closesection




\medskip
% =======================================================
\section{Articles in Conference Proceedings  (peer reviewed)}


% -------------------------------------------------------
  \cvbibitem 
    {CPU Ray Tracing Large Particle Data with Balanced P-k-d Trees}
    {Ingo Wald, Aaron Knoll, Gregory P Johnson, Will Usher, Valerio Pascucci, and Mike E Papka}
    {Proceedings of IEEE Visualization}{2015}

% -------------------------------------------------------
  \cvbibitem 
    {Efficient Ray Tracing of Subdivision Surfaces using Tessellation Caching}
    {Carsten Benthin, Sven Woop, Matthias Niessner, Kai Selgard, and Ingo Wald}
    {Proceedings of High Performance Graphics}{2015}

% -------------------------------------------------------
  \cvbibitem 
    {SIMD Parallel Ray Tracing of Homogeneous Polyhedral Grids}
    {Brad Rathke, Ingo Wald, Kenneth Chiu, and Carson Brownlee}
    {Eurographics Symposium on Parallel Graphics and Visualization (EGPGV)}{2015}

% -------------------------------------------------------
  \cvbibitem 
{Exploiting Local Orientation Similarity for Efficient Ray Traversal of Hair and Fur }{Sven Woop, Carsten Benthin, Ingo Wald, Gregory S Johnson, and Eric Tabellion} 
{High Performance Graphics 2014}{}

% -------------------------------------------------------
  \cvbibitem 
    {Ray Tracing and Volume Rendering Large Molecular Data on Multi-core and Many-core Architectures}
    {Aaron Knoll, Ingo Wald, Paul Navratil, Michael E Papka, and Kelly P Gaither}
    {Proc. 8th International Workshop on Ultrascale Visualization at SC13 (Ultravis)}{2013}

\cvbibitem{Extending a C like Language for Portable SIMD Programming}
{Roland Leissa, Sebastian Hack, and Ingo Wald}
{PPoPP '12: Proceedings of the 17th ACM SIGPLAN symposium on Principles and Practice of Parallel Programming 2012}
{}

% -------------------------------------------------------
\cvbibitem
{Full-Resolution Interactive CPU Rendering
with Coherent BVH Traversal}
{
Aaron Knoll, 
Sebastian Thelen,
Ingo Wald,
Charles D Hansen,
Hans Hagen,
Machael E Papka}
{Proceedings of IEEE Pacific Visualization 2011}
{}

% -------------------------------------------------------
\cvbibitem{Active Thread Compaction for GPU Path Tracing}
{Ingo Wald}
{Proceedings of High Performance Graphics 2011}
{}

% -------------------------------------------------------
\cvbibitem{Efficient Stack-less Traversal Algorithm for Ray Tracing with BVH}
{M Hapala, T Davidoviv, I Wald, V Havran, P Slusallek}
{Proceedings of SCCG 2011}
{pages 29--34}

% -------------------------------------------------------
\cvbibitem{Efficient ray traced soft shadows using multi-frusta tracing}
 {Carsten Benthin and Ingo Wald}
{In 
Proceedings of High-Performance Graphics 2009 (HPG09)}{}


% -------------------------------------------------------
\cvbibitem {Getting Rid of Packets -- SIMD Single-Ray Traversal using Multi-Branching BVHs}
{
Ingo Wald, Carsten Benthin, Solomon Boulos
} {In Proceedings of the 2008
  IEEE/Eurographics Symposium on Interactive Ray Tracing}{Los Angeles, 2008}

% -------------------------------------------------------
\cvbibitem {Adaptive Ray Packet Reordering}{
Solomon Boulos, Ingo Wald, Carsten Benthin
} {In Proceedings of the 2008
  IEEE/Eurographics Symposium on Interactive Ray Tracing}{Los Angeles, 2008}

% -------------------------------------------------------
\cvbibitem {Ray Tracing with the BSP Tree}{
Thiago Ize, Ingo Wald, Steven Parker
} {In Proceedings of the 2008
  IEEE/Eurographics Symposium on Interactive Ray Tracing}{Los Angeles, 2008}

% -------------------------------------------------------
\cvbibitem {On Fast Construction of SAH based Bounding Volume
  Hierarchies}{Ingo Wald} {In Proceedings of the 2007
  IEEE/Eurographics Symposium on Interactive Ray Tracing}{Ulm,
  Germany, 2007}


% -------------------------------------------------------
\cvbibitem {Interactive Ray Tracing of Arbitrary Implicits with SIMD Interval Arithmetic}
{Aaron Knoll, Charles D.\ Hansen, Younis Hijazi, Hans Hagen, \und Ingo Wald}{In Proceedings of the 2007 IEEE/Eurographics Symposium on Interactive
    Ray Tracing}{}


% -------------------------------------------------------
    \cvbibitem{Interactive Iso-Surface Ray Tracing of Massive
    Volumetric Data Sets}
{Heiko Friedrich, Ingo Wald, and Philipp Slusallek}{Eurographics Symposium on Parallel Graphics and
    Visualization '07}{}

% -------------------------------------------------------
    \cvbibitem {Asynchronous BVH Construction for Ray Tracing Dynamic
    Scenes}{Thiago Ize, Ingo Wald, \and Steven G.\ Parker}{Eurographics Symposium on Parallel Graphics and
    Visualization '07}{}

% -------------------------------------------------------
\cvbibitem {Packet-based Whitted and Distribution Ray Tracing}
{Solomon Boulos, Dave Edwards, J.\ Dylan Lacewell, Joe Kniss, Jan Kautz,
  Ingo Wald, \und   Peter Shirley}{Graphics Interface 2007}{(to appear)}

% -------------------------------------------------------
\cvbibitem {Interactive
  Isosurface Ray 
  Tracing of Large Octree Volumes}
{Aaron Knoll, Ingo Wald, Steven G.\ Parker, \und Charles D.\
  Hansen}{\proc the 2006 IEEE  Symposium on Interactive Ray Tracing}{
Salt Lake City, UT, 
%Sep~18--20, 
2006,
\pages~115--124}

% -------------------------------------------------------
\cvbibitem {Realtime Ray Tracing on the CELL Processor}
{Carsten Benthin, Ingo Wald, Michael Scherbaum, \und Heiko Friedrich}
  {\proc the 2006 IEEE  Symposium on Interactive Ray Tracing}{Salt
    Lake City, UT, Sep~18--20, 2006, \pages~15--23}

% -------------------------------------------------------
\cvbibitem {An Evaluation of Parallel Grid Construction for Ray Tracing Dynamic Scenes}
{Thiago Ize, Ingo Wald, Chelsea Robertson, \und Steven G Parker}
{\proc the 2006 IEEE  Symposium on Interactive Ray Tracing}{Salt
  Lake City, UT, Sep~18--20, 2006, \pages~47--55}

% -------------------------------------------------------
\cvbibitem{Applying Ray Tracing for Virtual Reality and Industrial
  Design}
{ Ingo Wald, Andreas Dietrich, Carsten Benthin, Alexander Efremov, Tim Dahmen, Johannes G\"unther,
  Vlastimil Havran, Hans-Peter Seidel, \und Philipp Slusallek}{
  \proc the 2006 IEEE Symposium on Interactive Ray Tracing}{Salt Lake City, UT, Sep~18--20, 2006,
\pages~177--185}

% -------------------------------------------------------
\cvbibitem{On building good KD-Trees for Ray
  Tracing, and on doing that in O(N~log~N)}{ Ingo Wald \und Vlastimil Havran}{
  \proc the 2006 IEEE  Symposium on Interactive Ray Tracing}{Salt Lake City, UT, Sep~18--20, 2006,
\pages~61--69}

% -------------------------------------------------------
\cvbibitem{An Application of
  Scalable Massive Model Interaction using Shared Memory Systems}
{ Abe Stephens, Solomon Boulos, \und Ingo Wald}{
  \proc EUROGRAPHICS Symposium on Parallel Graphics and Visualization}{Braga, Portugal, 
  11.--12.~\may 2006, \pages~19--26}

% -------------------------------------------------------
  \cvbibitem{
      Large-Scale CAD Model Visualization on a Scalable Shared-Memory
      Architecture}{ Andreas Dietrich, Ingo Wald, \und Philipp Slusallek}{\proc Vision, Modeling, and
    Visualization (VMV) 2005}{\pages~303--310, Erlangen, Germany, November 16--18, 2005}

% -------------------------------------------------------
\cvbibitem{{Efficient} {Acquisition} and {Realistic} {Rendering} of
  {Car} {Paint}}
{ Johannes G{\"u}nther, Tongbo Chen, Michael Goesele, Ingo Wald, \und Hans-Peter Seidel}
{\proc Virtual Reality, Modeling, and Visualization
  (VMV) 2005}{\pages~487--494, Erlangen, Germany, November 16--18, 2005}

% -------------------------------------------------------
\cvbibitem {{Interactive} {Ray} {Tracing} of
{Point} {Based} {Models}}
{Ingo Wald \und Hans-Peter Seidel}{Proceedings of 2005 Symposium on
Point Based Graphics}{
%Long Island, NY, USA, 
2005}

% -------------------------------------------------------
  \cvbibitem{
      Interactive Ray Tracing of Free-Form Surfaces}{ Carsten Benthin, Ingo Wald, \und Philipp Slusallek}{ \proc Afrigraph
      2004}{\pages 99--106, Stellenbosch, ZA, \ende{Nov 3--5}{3.--5.\ November},
      2004}

% -------------------------------------------------------
  \cvbibitem{An
      Interactive Out-of-Core Rendering Framework for Visualizing
      Massively Complex Models}
{ Ingo Wald, Andreas Dietrich, \und Philipp Slusallek}{In Rendering Techniques 2004, 
    EUROGRAPHICS Symposium on Rendering}{\pages~81--92, Norrk{\"o}ping,
      Sweden, \ende{June 21--23}{21.--23.
    Juni}, 2004}

% -------------------------------------------------------
  \cvbibitem {
      Realtime Caustics using Distributed Photon Mapping}{Johannes G\"unther, Ingo Wald, \und Philipp Slusallek}{\in
    Rendering Techniques 2004, EUROGRAPHICS Symposium on Rendering}{\pages~111--121
    Norrk{\"o}ping, Sweden, \ende{June 21--23}{21.--23. Juni}, 2004}

% -------------------------------------------------------
  \cvbibitem{VRML Scene Graphs on an Interactive Ray Tracing
      Engine}{ Andreas Dietrich, Ingo Wald, Markus Wagner, \und Philipp
    Slusallek}{\proc IEEE VR 2004}{ \pages~109--116, Chicago, USA,
    \ende{Mar 27--31,}{27.-31.
    März,} 2004}

% -------------------------------------------------------
\cvbibitem {Fast and Accurate Ray-Voxel Intersection Techniques for Iso-Surface
  Ray Tracing}{Gerd Marmitt, Andreas Kleer, Ingo Wald, Heiko Friedrich, \und Philipp Slusallek}{ \proc Virtual Reality, Modelling, and Visualization
    (VMV) 2004}{ \pages~429--435,  2004}


% -------------------------------------------------------
  \cvbibitem{
      Distributed Interactive Ray Tracing of Dynamic Scenes}{
 Ingo Wald, Carsten Benthin, \und Philipp Slusallek}{\in
    Proceedings of the IEEE Symposium on Parallel and Large-Data
    Visualization and Graphics (PVG)}{ \pages~77--86, Blaubeuren,
      Germany, \ende{Sep 9--10,}{9.-10.\ September,} 2003}


% -------------------------------------------------------
  \cvbibitem
    {Streaming Video Textures for Mixed Reality Applications in
      Interactive Ray Tracing Environments}
{ Andreas Pomi, Gerd Marmitt, Ingo Wald, \und Philipp Slusallek}{ \proc Vision, Modeling and
    Visualization (VMV) 2003}{ \pages~261-269, 
%Munich, 
Germany,
    \ende{Nov 19--21}{19.-21. November}, 2003}

% -------------------------------------------------------
  \cvbibitem{
      Interactive Global Illumination in Complex and Highly Occluded
      Environments}{ Ingo Wald, Carsten Benthin, \und Philipp Slusallek}{Proceedings of the 14th EUROGRAPHICS Workshop
      on Rendering}{P.~H.~Christensen and D.~Cohen-Or (editors),
      \pages~74--81, Leuven, Belgium, 2003}

% -------------------------------------------------------
  \cvbibitem{The
      OpenRT Application Programming Interface - Towards A Common API
      for Interactive Ray Tracing}{
 Andreas Dietrich, Ingo Wald, \und Philipp Slusallek}{Proceedings of the 2003 OpenSG
    Symposium}{\pages~23--31, Darmstadt, Germany, \ende{Apr 1--2}{1.--2. April}, 2003}


% -------------------------------------------------------
  \cvbibitem{SaarCOR
      - A Hardware Architekture for Ray Tracing}
{ J\"org Schmittler, Ingo Wald, \und Philipp Slusallek}{ \proc
    ACM SIGGRAPH/EUROGRAPHICS Graphics Hardware 2002}{\pages~27--36, Saarbrücken,
    Germany, \ende{Sep 1--2}{1.-2. September}, 2002}

% -------------------------------------------------------
  \cvbibitem{Interactive Headlight Simulation - A Case Study for
      Interactive Distributed Ray Tracing}
{ Carsten Benthin, Tim Dahmen, Ingo Wald, \und Philipp Slusallek.
    }{ \proc
    EUROGRAPHICS Workshop on Parallel Graphics and Visualization (PGV)}{ \pages~81--88,
    Seattle, WA, \ende{Oct 20--21}{20.--21.\ Oktober}, 2002}

% -------------------------------------------------------
  \cvbibitem{Interactive Global Illumination using
      Fast Ray Tracing}
{ Ingo Wald, Thomas Kollig, Carsten Benthin, Alexander Keller,
    \und Philipp Slusallek}{Proceedings of the 13th EUROGRAPHICS
    Workshop on Rendering}{P.~Debevec and S.~Gibson
  (editors), \pages~15--24, Pisa, Italy, 2002}



% -------------------------------------------------------
  \cvbibitem {
      Interactive Distributed Ray-Tracing of Highly Complex Models}
{Ingo Wald, Philipp Slusallek \und Carsten Benthin}{ Rendering Techniques 2001 / Proceedings of the EUROGRAPHICS
      Rendering Workshop 2001}{S.J.~Gortler \und K.~Myszkowski
      (editors), \pages~274--285, 
%London, UK, 
\ende{Jun}{Juni} 2001}

% -------------------------------------------------------
\cvbibitem{Efficient Importance Sampling Techniques
  for the Photon Map}{ Alexander Keller \und Ingo Wald}{\proc Vision
  Modelling and
 Visualization    (VMV)}{Saarbr\"ucken, Germany, \ende{Nov}{November} 2000}
%
\closesection





\newpage
\medskip
% =======================================================
\section{State of the Art Reports (peer reviewed)}

% -------------------------------------------------------
  \cvbibitem 
    {State of the Art in Ray Tracing Animated Scenes}{Ingo Wald,
    William R Mark, Johannes G\"unther, Solomon Boulos, Thiago Ize,
    Warren Hunt, Steven G.\ Parker, and Peter Shirley}{EUROGRAPHICS
    2007 State of the Art Reports}{Prague, Czech Republic, 2007}


% -------------------------------------------------------
  \cvbibitem 
    {Realtime Ray Tracing and its use for Interactive Global
      Illumination}{Ingo Wald, Timothy J.\
    Purcell, J\"org Schmittler, Carsten Benthin, \und Philipp Slusallek}{EUROGRAPHICS 2003 State of the Art Reports}{Granada, Spain, 2003}


% -------------------------------------------------------
  \cvbibitem {State-of-the-Art in Interactive Ray
      Tracing}{Ingo Wald \und
    Philipp Slusallek}{ EUROGRAPHICS 2001 State of the Art Reports}{EUROGRAPHICS,
%Manchester, UK, 
2001}

\closesection


\medskip
%\pagebreak
% =======================================================
\section{Refereed Tutorials and  Courses at Conferences}

% -------------------------------------------------------
\cvbibitem{ACM SIGGRAPH 2013 Course on the Future of Ray Tracing}{organized by Alexander Keller, NVidia}{ACM
  SIGGRAPH 2013}{(to appear)}

% -------------------------------------------------------
\cvbibitem{ACM SIGGRAPH 2006 Course on Realtime Ray Tracing}{\mit Philipp
  Slusallek, Peter Shirley (University of Utah), Bill Mark (University
  of Texas), Gordon Stoll (INTEL), \und Dinesh Manocha (UNC)}{ACM
  SIGGRAPH 2006}{Boston, USA, 2006}

% -------------------------------------------------------
\cvbibitem{ACM SIGGRAPH 2005 Course on Realtime Ray Tracing}{\mit Philipp
  Slusallek, Peter Shirley (University of Utah), Bill Mark (University
  of Texas), \und Gordon Stoll (INTEL)}{ACM SIGGRAPH 2005}{Los Angeles, USA, 2005}

% -------------------------------------------------------
\cvbibitem{Afrigraph 2004 Course on Massive Model Visualization}{\mit
  Andreas Dietrich}{Afrigraph 2004}{Stellenbosch, South Africa, 2004}


% -------------------------------------------------------
\cvbibitem{ACM SIGGRAPH 2003 Course on Global Illumination for High-Quality
  Animations and Interactive Applications}{\mit Karol
  Myszkowski, Cyrille Damez, Per H. Christensen, Bruce Walter, \und
  Philipp Slusallek}{ACM SIGGRAPH 2003}{
Los Angeles, 
USA, 
2003}

% -------------------------------------------------------
\cvbibitem{Afrigraph 2003 Course on Advanced Issues in Realtime Ray
  Tracing and Interactive Global Illumination}{\mit Carsten
  Benthin}{Afrigraph 2003}{Cape Town, South Africa, 2003}


% -------------------------------------------------------
\cvbibitem{Afrigraph 2001 Course on Interactive Ray Tracing}{Ingo
  Wald}{Afrigraph 2001}{Cape Town, South Africa, 2001}


\closesection


\medskip
% =======================================================
\section{Masters and Doctoral Theses}

% -------------------------------------------------------
\cvbibitem {Realtime Ray Tracing and Interactive Global
  Illumination}{Ingo Wald}{PhD thesis}{Saarland University, 2004
% (available at
%  http://www.sci.utah.edu/~wald/PhD). 
\ende{(Grade: with distinction)}{\\ Note: mit Auszeichnung}}

% -------------------------------------------------------
\cvbibitem{Photorealistic Rendering using the Photon Map}{Ingo Wald}{
  Diplomarbeit\ende{ (masters thesis)}{}}{Universit\"at
  Kaiserslautern, 1999. Grade: sehr gut (A equivalent)}

\closesection








%\medskip
%% =======================================================
%\section{Articles currently submitted for peer review}


% -------------------------------------------------------
%\cvbibitem{Fast Ray Tracing of Catmull-Clark Subdivision Surfaces}
%{Carsten Benthin, Solomon Boulos, Jesse Dylan Lacewell, and Ingo Wald}{}{}

% -------------------------------------------------------
%\cvbibitem{Exploring a Boeing~777 -- Ray Tracing Large-Scale CAD Data}
%{ Andreas Dietrich, Abe Stevens, Ingo Wald, Dave Kasik, \and Philipp Slusallek}{}{}


%% -------------------------------------------------------
%\cvbibitem {Interactive Ray Tracing of Arbitrary Implicits}
%{Aaron Knoll, Charles D.\ Hansen, Younis Hijazi, Hans Hagen, \und Ingo Wald}{}{
%\meanwhileavailable SCI Institute, University of Utah, Technical Report
%No UUSCI-2007-002}



%\medskip
\newpage
% =======================================================
\section{Technical Reports (if not mentioned otherwise)}

\cvbibitem{Screen-Space Spherical Harmonics Filters for Instant Global Illumination}
{Benjamin Segovia and Ingo Wald}{Intel Corp}{2009}

\cvbibitem{Fast Ray Tracing of Catmull-Clark Subdivision Surfaces}
{Carsten Benthin, Solomon Boulos, Jesse Dylan Lacewell, and Ingo Wald}{SCI Institute, University of Utah}{2007}

\cvbibitem{SIMD Stream Tracing --- SIMD Ray Traversal with Generalized Ray Packets
and On-the-fly Reordering}{Ingo Wald}{SCI Institute, University of Utah}{2007}

% -------------------------------------------------------
%\cvbibitem{Coherent Multiresolution Isosurface Ray Tracing}
%{ Aaron Knoll, Charles D Hansen, \und Ingo Wald}{}{
%available SCI Institute, University of Utah, Technical Report
%No UUSCI-2007-001} 

% -------------------------------------------------------
\cvbibitem{Geometric and Arithmetic Culling Methods for Entire Ray
Packets}{  Solomon Boulos, Ingo Wald, \und Peter Shirley}{Technical
  Report No.\ UUCS-06-010}{School of Computing, University of Utah, 2006}

% -------------------------------------------------------
\cvbibitem{Interactive Distribution Ray Tracing.}
{ Solomon Boulos, David Edwards, Jesse Dylan Lacewell, Joe Kniss, Jan
  Kautz, Peter Shirley, \und Ingo Wald}
{
Technical Report No.\ UUSCI-2006-022}{SCI Institute, University of Utah, 2006}

% -------------------------------------------------------
\cvbibitem{Ray Tracing
    Deformable Scenes using Bounding Volume Hierarchies}
{ Ingo Wald, Solomon Boulos, \und Peter Shirley}{Technical Report No.\ UUSCI-2006-023}{SCI Institute,
  University of Utah. 2006.  (more detailed version
  of \emph{ACM Transactions on
    Graphics}, 26 (1), 2007)}


\cvbibitem {Discretized Incident Radiance Maps for Interactive
  Global Illumination in Complex Environments}{Ingo Wald}{Technical Report No.\ UUSCI-2005-010}{SCI Institute, University of Utah. 2005}


\cvbibitem {Towards Realtime Ray Tracing---Issues and Potential}
{Ingo Wald, Carsten Benthin, \und Philipp Slusallek}{Technical Report}{
Saarland University. 2003}

\cvbibitem {{OpenRT}---{A} {Flexible} and {Scalable} {Rendering}
 {Engine} for {Interactive} {3D} {Graphics}} {Ingo Wald, Carsten
 Benthin, \und Philipp Slusallek}{Technical Report}{Saarland
 University. 2002}
% available at http://graphics.cs.uni-sb.de/Publications


\cvbibitem{Interactive Raytracing on Notebooks}{Ingo Wald, Philipp Slusallek, Carsten Benthin, \und Markus Wagner}{Technical Report}{Saarland
 University. 2002}

\closesection




\medskip
% =======================================================
\section{ACM SIGGRAPH Sketches and Tech Talks, peer-reviewed}

% -------------------------------------------------------
\cvbibitem{Embree 2.0: A Real-time Ray Tracing Infrastructure for Xeon and Xeon PHI}{With Louis Feng, Sven Woop, and Carsten Benthin}{ACM SIGGRAPH Tech Talks (peer reviewed)}{2013}{}

% -------------------------------------------------------
\cvbibitem{{Interactive} {Ray} {Tracing} of
  {Point} {Based} {Models}}{ Ingo Wald \und Hans-Peter Seidel}{ACM SIGGRAPH Sketches and
  Applications}{2005}

% -------------------------------------------------------
\cvbibitem {{Precomputed} {Light} {Sets} for {Fast}
  {High} {Quality} {Global} {Illumination}}{Johannes G\"unther, Ingo Wald, \und Hans-Peter Seidel}
   {ACM SIGGRAPH Sketches and Applications}{2005}

% -------------------------------------------------------
\cvbibitem {{Interactive}
  {Visualization} {of} {Exceptionally} {Complex} {Industrial}
  {Datasets}}{Andreas Dietrich, Ingo Wald, \und Philipp Slusallek}{{ACM} {SIGGRAPH} {Sketches} {and}
    {Applications}}{2004}


\closesection


\medskip
% =======================================================
\section{Other Publications and Invited Papers}

% -------------------------------------------------------
\cvbibitem{SIMD Stream Tracing -- SIMD Ray Tracing with Generalized
  Packets and On-the-fly Re-ordering}
{ Ingo Wald}{}{Poster at the 2007 IEEE/EG Symposium on Interactive Ray Tracing}

% -------------------------------------------------------
\cvbibitem {Realtime Ray Tracing and Interactive Global
  Illumination}{Ingo Wald}
{it--In\-for\-ma\-tion Technology}{ Nummer 4, 2006. \indeutsch}

% -------------------------------------------------------
\cvbibitem {Ray Tracing Animated Scenes--One Experiment and Three
  Solutions}{Ingo Wald}
{The Utah Teapot}{Fall 2006}

% -------------------------------------------------------
\cvbibitem{{Realtime} {Ray} {Tracing} and {Interactive} {Global}
  {Illumination}}{Ingo Wald}{Aus\-ge\-zeich\-ne\-te Informatikdissertationen
  2005}{Bonner K{\"o}llen Verlag, Dorothea Wagner et al.(eds), series
  GI-Edition Lecture Notes in Informatics (LNI), 2005. \indeutsch}

%\cvbibitem{inView/OpenRT Installation Guide and User Manual}{Ingo Wald
%  and inTrace GmbH}{}{2002--2006}

% -------------------------------------------------------
\cvbibitem {Interaktive Beleuchtungssimulation
 und Bildsynthese mit Ray-Tracing -- Effiziente Software schl\"agt
 Spezial-Hardware}{Philipp Slusallek \und Ingo Wald}{Magazin For{\-}schung 1/2002 der Universit\"at
 des Saarlandes}{ISSN 0937-7301, pages 2--12, \ende{May}{Mai} 2002 \indeutsch}

\closesection






%\pagebreak
% =======================================================
% =======================================================
% =======================================================
\bigskip
\bigskip
\bigskip
\firstnamestyle{\Huge{Invited Talks and Presentations (selected)}}
\vspace*{6mm}

\section{}

\cvownitem{Embree---A Ray Tracing Kernel Framework for Efficient CPU Ray Tracing}  {Graphics Seminar, University of Texas in Austix}{2014}

\cvownitem{Ray Tracing for Interactive Rendering and Visualization:
  Are we done, yet?}  {Keynote at the Eurographics Conference on
  Parallel Graphics and Visualization (EGPGV) 2013}{}

% -------------------------------------------------------
\cvownitem{IVL--A experimental SPMD Compiler for SSE and MIC}
{Intel VPG Tech Summit (MIC)}{Jan 2012, (best presentation award)}
% -------------------------------------------------------
\cvownitem{Active Thread Compaction for GPU Path Tracing}
{Intel OpenCL Forum}{2011}
% -------------------------------------------------------
\cvownitem{IVL+R(IVL)--A experimental SPMD Compiler for SSE, AVX, and MIC, and its application to Real-Time Realistic Rendering}
{Illinois-Intel Parallelism Center (I2PC) at the University of Illinois Urbana Champaign (UIUC)}{Oct 20, 2011}
% -------------------------------------------------------
\cvownitem{Fast Ray Tracing on LRB using Multi-Packet/Multi-Frustum Traversal}
{Intel UPCRC Summit}{Oregon 2009 (best talk award)}
% -------------------------------------------------------
\cvownitem{Aggressive Packet-/Frustum Traversal Techniques -- State of
the Art and Current Issues}
{Eurographics IPC Minisymposium}{Prague, Czech Republic. April 2007}
% -------------------------------------------------------
\cvownitem{On ray tracing and future graphics architectures -- and how
these two will influence each other}
{NVidia Corp}{Santa Clara, CA, USA. Feb 2007}
% -------------------------------------------------------
%\cvownitem{Echtzeit Ray Tracing}{Universit\"at
%  Magdeburg}{Germany, Dez 2006 \indeutsch}
%
% -------------------------------------------------------
%\cvownitem{Realtime Ray Tracing}{University of Bath}{Bath, UK, Nov 2006}
%
%\vspace*{-1em}
%
% -------------------------------------------------------
\cvownitem{Realtime Ray Tracing}{IBM TJ Watson}{Yorktown
  Heights, 
%NY,
 Oct 2006}
%
\vspace*{-1em}
%
% -------------------------------------------------------
\cvownitem{Realtime Ray Tracing}{University of Wales in
  Bangor}{Bangor, Wales, UK, Sep 2006}
%
\vspace*{-1em}
%
% -------------------------------------------------------
\cvownitem{Ray Tracing Dynamic Scenes -- Recent Progress, Current
  Issues}{Intel Corp}{ Santa Clara, CA, USA, 2006}
%
% -------------------------------------------------------
%% \cvownitem{Realtime Ray Tracing -- Progress and Issues}{
%%   \ende{Research colloquium,}{Forschungskolloquium der} University of Utah}{Salt Lake City, UT, USA, 2006}
%
% -------------------------------------------------------
\cvownitem{Recent Trends in Realtime Ray Tracing -- Practical
  Applications, Complex Models, and Animated Scenes}{
\ende{Invited Talk at the annual meeting of the advisory board of
  german university professors}{
  Gastvortrag beim GIBU-Jahrestreffen (GI Beirat der
  Universit\"ats\-professoren)}}{
%Dagstuhl, \ende{Germany,}{} 
2006}
%
% -------------------------------------------------------
\cvownitem{Ray Tracing Animated Scenes}{MPI f\"ur Informatik}{
   Saarbr\"ucken, Germany, 2006}
%
\vspace*{-1em}
%
% -------------------------------------------------------
\cvownitem{Realtime Ray Tracing and Interactive Global Illumination}{
  \ende{Finalists round of the BMW Innovation Award}{Finalistenrunde des BMW Innovationspreises}}{BMW FIZ, M\"unchen, Germany, 
  2005}
%
% -------------------------------------------------------
\cvownitem{Realtime Ray Tracing and Interactive Global Illumination}{
  \ende{Finalists round of the GI Dissertationspreis (Dissertation
  Award of the German Association of Computer Scientists)}{Finalistenrunde des GI Dissertationspreises}}{Mainz, Germany, 2005}
%
%
%% % -------------------------------------------------------
%% \cvownitem ``Interaktive Visualisierung eines hoch komplexen Boeing 777
%%   Modells'', Andreas Dietrich and Ingo Wald, Cover Image and Short
%%   Article on Informatik Spektrum 22(4), pages 394--394, August 2004
%
%% % -------------------------------------------------------
%% \cvownitem Cover Image on Proceedings of Parallel Graphics and Visualization 2003
%
% -------------------------------------------------------
\cvownitem{Realtime Ray Tracing and Interactive Global Illumination}{
%  Gastvortrag, 
Stanford
% University
}{%Palo Alto, CA, 
2002}
%
% -------------------------------------------------------
\cvownitem{The Saarland RTRT/OpenRT Realtime Ray Tracing Project}{Intel Corp}{%
%Santa Clara, CA, 
2002}
%
% -------------------------------------------------------
\cvownitem{Towards interactive Global Illumination}{Dagstuhl
  Seminar}
{%
%Germany, 
2001}
%
\cvownitem{Photorealistic Rendering using the Photon Map}{Dagstuhl
  Seminar}{%
%Dagstuhl, Germany, 
2000}
%
\closesection


%\pagebreak
% =======================================================
% =======================================================
% =======================================================
%\firstnamestyle{\Huge{Public Relations (selected)}}
%\vspace*{6mm}

%\section{}
%\closesection


%\pagebreak
% =======================================================
% =======================================================
% =======================================================
\bigskip
\bigskip
\bigskip
\firstnamestyle{\Huge{Awards}}

\vspace*{6mm}

\section{}

\cvownitem{HPG'15 ``Test of Time 2006'' Award}{(1st place)}{ 
%High  Performance Graphics 2015, 
for Wald and Havran, \emph{On Building
    kd-Trees for Ray Tracing, and on doing that in O(N log N)}}{}

\cvownitem{HPG'15 ``Test of Time 2006'' Award}{(2nd place)}{ 
%High Performance Graphics 2015, 
for Benthin et al, \emph{Ray Tracing on
    the CELL processor}}{}

\cvownitem{HPG'15 ``Wolfgang Strasser'' Best Paper Award}{}{ 
%High Performance Graphics 2015, 
for Benthin et al, \emph{Efficient Ray Tracing of Subdivision
    Surfaces using Tessellation Caching}}{}

\cvownitem{Intel Technical Computing Group (TCG) Recognition Award}{}{For the 
creation and public release of Embree 2.0, Q3-2013}{}

\cvownitem{Intel DRD Divisional Recognition Award}{SSG/DCSG (Software Solutions Group/Datacenter and Services Group)}{SC12 Keynote Demo of Dreamworks Xeon Phi App, Q1-2013}{}

\cvownitem{Intel Divisional Recognition Award}{Microprocessor and Programming Research}{to Ingo Wald, For developing Embree 2.0 for Xeon Phi, demonstrating superior performance for photo-realistic rendering, transforming Dreamworks' lighting engine from tens of minutes-per-frame to near real-time, Q4-2012}{}

\cvownitem{Best Presentation, 2nd prize -- Intel VCG Tech Summit 2012}
{In recognition of the Best Presentation during the Intel VCG (Visual Computing Group) Tech Summit -- Spring 2012}{}{}{}

\cvownitem{Intel Divisional Recognition Award}{Microprocessor and Programming Research Lab}{to Ingo Wald, for Demonstrating Intel Graphics Leadership with the first public live Advanced Graphics Demonstration on Larrabee, Q1-2010}{}

\cvownitem{MIC (Larrabee) - Committed to a new Parallel Computing Roadmap}{to Ingo Wald}{Mai 2010}{}

\cvownitem{Best Demo Award -- Intel VCG Tech Summit}
{In recognition of the Best Demo during the Intel VCG (Visual Computing Group) Tech Summit -- Fall 2009}{}{}{}


\cvownitem{Best Talk -- Intel UPCRC Summit}{for a talk on \emph{Fast
    Ray Tracing on LRB using Multi-Packet Traversal}}{Beaverton, OR,
  September 2009}{}


\cvownitem{SaarLB Wissenschaftspreis 2004\ende{ (SaarLB Science Award
      2004)}{}}
{Science award of the Saarl\"andische Landesbank (endowed with
25,000 Euros)}{for the dissertation \emph{Realtime Ray Tracing and
Interactive Global Illumination}, 2005}

\cvownitem{\ende{Finalists round of the BMW Innovation Award (5
    Finalists)}{Endrunde des BMW Innovation Award (5 Finalisten)}}{BMW Group}
{M\"unchen, July 2005,  \emph{Realtime Ray Tracing and Interactive Global Illumination}}
%\cvownitem At Eurographics 2001 Conference, Manchester, GB, for the paper \textit{``Interactive Rendering with Coherent Ray Tracing''}.

\cvownitem{Finalists round of the GI Dissertation Award 2004}
{(Dissertation award of the German Association of Computer
    Scientists)} 
{Mainz, Germany, 2005}

\cvownitem{Dr.-ing. ``mit Auszeichnung''\ende{ (PhD ``with
    distinction'')}{}}
{Universit\"at des Saarlandes}
{
\ende{For the dissertation}{f\"ur die Dissertation}
  \emph{Realtime Ray Tracing and Interactive Global Illumination}}
%\cvownitem At Eurographics 2001 Conference, Manchester, GB, for the paper \textit{``Interactive Rendering with Coherent Ray Tracing''}.

\cvownitem{Best paper award}{(2nd place)}{
 Eurographics 2001, \ende{for}{f\"ur} \emph{Interactive Rendering with
 Coherent Ray Tracing}\ende{ (with over 500 scholar.google.com
 citations to date)}{}}

\cvownitem{HPCwire Editor's Choice Award}{awarded to the OpenRT Realtime Ray Tracer}{}{}
%
\cvownitem{European Innovation Prize}{awarded to inTrace GmbH}{}{}

\closesection



\bigskip
\bigskip
\bigskip


%\pagebreak
% =======================================================
% =======================================================
% =======================================================
%\firstnamestyle{\Huge{Public Events}}

%\vspace*{6mm}

%r
%\section{TODO}

%\cvownitem{Intel Developer Forum 2009}{Ray Tracing Demo at IDF 2009}{September 2009}{}{}

%to add here: idf demo, boeing demo, idf gelsinger demo, vw event

%\closesection

\bigskip
\bigskip
\bigskip



\pagebreak
% =======================================================
% =======================================================
% =======================================================

\firstnamestyle{\Huge{Professional Service}}
\vspace*{6mm}

\section{Chair/Organizer Duty}
\cvlistitem{
Papers chair, High-Performance Graphics 2014}
\cvlistitem{
Papers chair, High-Performance Graphics 2009}
\cvlistitem{
Conference Chair, IEEE Symposium on Interactive Ray Tracing 2006}
%
%\closesection
%
%\section{Papers Co-Chair}
%
%\closesection
%
% =======================================================
%\section{Co-organizer}
%
% -------------------------------------------------------
\cvlistitem{
Co-Organizer, ACM SIGGRAPH 2006 Course on Interactive Ray Tracing
}
\closesection

% ``finance chair'' for RT07 ?

% =======================================================
\section{Committees}

% -------------------------------------------------------
\cvlistitem{
Intel Labs Patent Committee
}
\cvlistitem{
Selection Committee for the4 ``Intel Sponsored Sessions'' at ACM SIGGRAPH
}
\cvlistitem{
Steering Committee, High Performance Graphics
}
\cvlistitem{
Steering Committee, EG/ACM Symposium on Interactive Ray Tracing
}
\cvlistitem{
Intel University Program Office (Implementation Task Force for the
European University Programme - 'til 2013)
}
\closesection

% =======================================================
\section{\ende{Editorial Boards}{Editorial Boards}}

\cvlistitem{
Computer Graphics Forum
%The Visual Computer 
}
\cvlistitem{
Computers and Graphics
%The Visual Computer 
}
\closesection


% =======================================================
\section{Programme Committees (selected)}

% -------------------------------------------------------
\cvlistitem{
 High-Performance Graphics 2009--current
}
% -------------------------------------------------------
\cvlistitem{
Eurographics Symposium on Parallel Graphics and Visualization 2008--current
}
% -------------------------------------------------------
\cvlistitem{
Graphics Hardware 2008
}
% -------------------------------------------------------
\cvlistitem{
Eurographics 2007, 2009, 2010
}
% -------------------------------------------------------Afrigr
\cvlistitem{
Afrigraph 2004, 2007
}
% -------------------------------------------------------
\cvlistitem{
 IEEE Symposium on Interactive Ray Tracing 2006, 2007, 2008
}
% -------------------------------------------------------
\cvlistitem{
International Symposium on Visual Computing (ISVC) 2007, 2008
}
% -------------------------------------------------------
\cvlistitem{
Computer Graphics and Visualization 2007 (CGV2007)
}
\closesection

% =======================================================
\section{Reviewer for (selected)}

\cvlistitem{
ACM Transactions on Graphics
}
\cvlistitem{
 IEEE Transactions on Visualization and Computer Graphics
}
\cvlistitem{
IEEE Computer Graphics and Applications
}
\cvlistitem{
 Computer Graphics Forum
}
\cvlistitem{
 ACM SIGGRAPH
%Conference,
}
\cvlistitem{
 Eurographics Symposium on Rendering (EGSR)
}
\cvlistitem{
 IEEE Visualization
}
\cvlistitem{
 Eurographics
}
\cvlistitem{
 Virtual Reality, Modeling, and Visualization (VMV)
}
\cvlistitem{
 Graphics Interface
}
\cvlistitem{
 Pacific Graphics
}
\cvlistitem{ 
Central European Journal of Physics
}
\cvlistitem{ 
International Symposium on Visual Computing (ISVC)
}
\closesection

% =======================================================
%% \section{Memberships}

%% % -------------------------------------------------------
%% \cvlistitem{
%% IEEE Visualization and Graphics Technical Committee (vgtc)
%% }
%% % -------------------------------------------------------
%% \cvlistitem{
%% ACM Siggraph
%% }
%% \cvlistitem{
%% Afrigraph (founding member)
%% }
%% \closesection










\pagebreak
% =======================================================
% =======================================================
% =======================================================

\firstnamestyle{\Huge{Grant Capture}}
\vspace*{-8mm}

~\\
\begin{center}
{%\color{sectionrectanglecolor}
\textbf{Note: Grant capture applies only between mid 2006 (when I became a research professor eligible to submit proposals) and mid 2007 (when I joined Intel).}}
\end{center}

\section{Grants awarded}

\cvownitem{OSPRay: A Open, Scalable, and Parallel Ray Tracing Infrastructure}
{Intel Visual Computing Institute Research Project Proposal}{PI: Ingo Wald, Intel Corp. Co-PIs: Prof Dr.-ing.\ Philipp Slusallek, Saarland University, requested amount 3xEUR~75,000 (granted Feb 2012)}{}
%
\cvownitem{SDCI HPC: Improvement and Release of the Uintah computational framework}
{NSF proposal number OCI-0721659}{PI: Martin Berzins, University of Utah. Co-PIs: Steven G.\ Parker and Ingo Wald,  University of Utah. Requested amount: \$920,495 (awarded to M.\ Berzins \& S.\ Parker after I left the University of Utah)}{}
%
%% \cvownitem{Hardware grant of various High-End CPUs and Workstations}
%% {Bulk hardware donation to equip PhD students at the University of Utah}{(value unknown)}{}
%
\cvownitem{Consulting Agreement}
{Intel Corp, \$50,000}{Summer 2007}{}
%
\closesection

\section{Applications Cancelled when I left Academia}

 \cvownitem{Intel Systems and Communications Research Council}{Realtime
   Ray Tracing of Advanced Lighting Effects using modern Multi-core
   Architectures}{PI: Ingo Wald, University of Utah.  Requested amount:
   \$225,000 over three years, plus equipment (positively reviewed,
   then cancelled due to conflict-of-interest once I joined
   Intel)}

%\cvownitem{ U.S.\ Army Research Laboratories}{Development of advanced ray tracing applications for ballistic
%penetration and electromagnetic propagation studies}{
%PI: Ingo Wald, University of Utah.
%Requested amount for the first year: \$150,000}

%% \section{Proposals Submitted but Not Awarded}

%% \cvownitem{
%% NSF proposal number  CCF-0702009}{Algorithms for Handling Advanced Lighting
%% Effects in Packet/Frustum-Based Ray Tracing Architectures}{ PI: Ingo
%% Wald, University of Utah. Requested amount: \$577,479}

\closesection

%Various hardware grants (Intel, AMD, NVidia) and loans (NVidia, IBM)
%awarded during my PhD and Post-doc time are not separately listed.

\bigskip
\bigskip
\bigskip


\vspace*{-8mm}
\firstnamestyle{\Huge{Industrial Experience}}
%\vspace*{-6mm}
%\vspace*{6mm}

\section{}

%\cventry{2000--2004}{
%Dr.-ing.%
%%PhD in Computer Science
%}{Saarland University,
%  Saarbr\"ucken, Germany}
%{}{}{Awarded Degree: Dr.-ing. mit Auszeichnung (PhD in Computer Science, with distinction)%}

\cventry{Jul 2013--current}{Intel Corp}{Technical Computing Group / Data Center Group}{Austin, TX}{}{Senior Staff Research Scientist, Tech Lead Ray Tracing and Software Defined Visualization}

\cventry{Feb 2013--Jun 2013}{Intel Corp}{Experiences Technology Lab, Intel Labs}{Austin, TX}{Senior Staff Research Scientist}{}

\cventry{Mar 2012--Feb 2013}{Intel Corp}{Intel Labs, Visual Applications Research Lab (VAR)}{Staff Research Scientist, attached to the Intel Visual Computing Institute, Saarbruecken, Germany}{}{}

\cventry{Nov 2007--Mar 2012}{Intel Corp}{Corporate Technology Group/Intel Labs, Research Scientist Visual Applications Research Lab (VAR)}{}{}{}
%\cventry{Nov 2007--current}{Intel Corp}{Corporate Technology Group, Visual Applications Research Lab (VAR)}{}{}{Senior Research Scientist}

\cventry{May--Aug 2007}{Intel Corp}{Santa Clara, CA,
  USA}{}{}
{Visiting Professor / Consultant in the Advanced Graphics Architectures Lab.}

\cventry{2003--2006}{\emph{in}Trace Realtime Ray Tracing
  GmbH}{Saarbr\"ucken, Germany}{}{}
{\ende{Co-founder and Vice-CEO.}{Mitgr\"under und zweiter
    Gesch\"aftsf\"uhrer}}

\cventry{May--Aug 2002}{Intel Corp}{Santa Clara, CA, USA}{}{}
{Internship, Microprocessor Research Labs / Advanced Graphics
  Architectures.}

%% \cventry{Feb--Mar 1997}
%% {Daimler-Benz Aerospace (DASA)}{Bremen, Germany}{(now EADS)}{}
%% {Internship, Division of Space \& Infrastructure, DASA, Bremen, Germany.
%% }
\closesection



%\cvownitem{NSF proposal number OCI-0721659}{SDCI HPC: Improvement and
%  Release of the Uintah computational framework.}{ }
%{PI: Martin Berzins, University of Utah. Co-PIs: Steven G.\ Parker and Ingo Wald,  University of Utah. Requested amount: \$920,495 (awarded, transferred to Steven Parker as sole PI after I left Utah)}

%\cvownitem{Intel}{Hardware grant of various High-End CPUs and Workstations
%  to equip PhD students at the University of Utah}{(bulk hardware
%  donation, value unknown)}



%\closesection
%\medskip
% =======================================================
%\section{Pending Funds}

% \cvownitem{Intel Systems and Communications Research Council}{Realtime
%   Ray Tracing of Advanced Lighting Effects using modern Multi-core
%   Architectures}{PI: Ingo Wald, University of Utah.  Requested amount:
%   \$225,000 over three years, plus equipment (positively reviewed,
%   but then blocked due to conflict-of-interest issues after I joined
%   Intel)}

%\cvownitem{ U.S.\ Army Research Laboratories}{Development of advanced ray tracing applications for ballistic
%penetration and electromagnetic propagation studies}{
%PI: Ingo Wald, University of Utah.
%Requested amount for the first year: \$150,000}

%\cvownitem{
%NSF proposal number  CCF-0702009}{Algorithms for Handling Advanced Lighting
%Effects in Packet/Frustum-Based Ray Tracing Architectures}{ PI: Ingo
%Wald, University of Utah. Requested amount: \$577,479}

%\cvownitem{NVidia professor partnership program}{Realtime Ray Tracing on a
%G80}{PI: Ingo Wald, University of Utah. Requested amount: \$36,000}



%\closesection









\pagebreak

%\pagebreak
% =======================================================
% =======================================================
% =======================================================

\firstnamestyle{\Huge{Research Projects I am/was Involved In (selected)}}
\vspace*{6mm}


\def\cvproject#1#2#3#4{\cvitem{{#1}}{{\bfseries
      #2.}{\ifthenelse{\equal{#3}{}}{}{\ #3}}{\ifthenelse{\equal{#4}{}}{}{\ \emph{#4}.}}}}

%\def\cvproject#1#2#3#4{\cvitem{{#1}}{{\bfseries
%      #2.\ }{\ifthenelse{\equal{#3}{}}{}{\\#3}}{\ifthenelse{\equal{#4}{}}{}{\\\emph{#4}.\ }}}}


% =======================================================
\section{}%Conducted Research Projects}

\cvproject{2013--ongoing}{OSPRay---A Open, Scalable, and Portable Ray
  Tracing Infrastructure}{}{Development of a new ray tracing based
  software renderer for scalable and high-fidelity visualization
  (early prototype)}

\cvproject{SC 2013}{Intel ``Marquee'' demo at SuperComputing 2013}{}{The Marquee demo at SC13 showcased several ray tracing vis applications (using Embree and RIVL/BNSView) on the ``Cherry Creek'' supercomputer (then \#400 on the Top500 list)}

\cvproject{2013--}{Ray Tracing based Visualization on Intel Architectures at TACC}{}{In close collaboration with the Texas Advanced Computing Center (TACC), funded by Intel Technical Computing Group and building on certain software packages I am tech lead on.}

\cvproject{SC 2012}{Xeon Phi/KNC Launch Demo at Supercomputing
  2012}{Using the Dreamworkd ``torch'' interactive lighting tool to
  showcase the launch of the Intel Xeon Phi product line at
  Supercomputing 2012 in Salt Lake City, UT.}{In collaboration with
  Dreamworks, the Intel Embree Team, and a team in Intel SSG.}

\cvproject{2012--}{Intel-Dreamworks Rendering Collaboration} {As part
  of an ongoing collaboration with Dreamworks, we are developing
  several technologies based on fast ray tracing on intel hardware,
  all using Embree 2.0: The Dreamworks ``torch'' interactive lighting
  preview tool, ray tracing of hair, ...}{In collaboration with
  Dreamworks, the Intel Embree Team, and a team in Intel SSG.}

\cvproject{2012--}{Embree 2.0}{A (originally IVL-based) version of the Intel
\emph{Embree} ray tracer that now embraces SPMD, also now runs on AVX and MIC/Xeon Phi.}{}

\cvproject{2012--}{(R)IVL}{A (IVL-based) portable and extensible
  high-performnace ray tracing architecture for Intel CPUs, with support for
  compressed large models, isosurfaces, volume rendering, balls-and-sticks, etc. Demo'ed,
  for example, at the Intel VCG Tech Summit (best presentation award),
  Research at Intel Europe Day, etc.}{Main author: Ingo Wald, in
  collaboration with Carsten Benthin.}

\cvproject{2011--}{ISPC}{The Intel SPMD Compiler Project (now available as open
  source at https://ispc.github.org).}{Main Author: Matt Pharr, Intel
  Corp}

\cvproject{2011--}{IVL}{A SPMD Program Compiler for Intel CPUs (with
  back-ends for SSE, AVX, and MIC/Knights), allowing to program Intel
  CPUs in a way similar to CUDA, while explicitly exposing the
  advantages of CPUs.}{Main Author: Ingo Wald; underlying theoretical
  part in collaboration with Hack et al, Saarland University}

%\cvproject{2010}{Denali}{A photo-realistic renderer infrastructure for
%Intel CPUs. Since released in Open Source under the name of ``Embree'')}{Main Authors: S.\ Woop, M.\ Ernst}

\cvproject{2010--}{Programming models research and
  evaluation}{Evaluating various programming models (intrinsics, AN,
  Elemental Functions, OpenCL, IVL/ISPC, \dots) in terms of both
  effectiveness and performance on various hardware architectures
  (integrated graphics, (GP)GPU, CPU, MIC, \dots).}{}

\cvproject{2008--2010}{Garfield}{A Intel-MIC (Larrabee/Knights)-based
  real-time ray tracing engine for game-like content. Demoed at
  various public events like Intel IDF, CeBit, etc.}{}

\cvproject{2007--2008}{RTTL (The Ray Tracing Template Library)}{A real-time ray tracing engine for
  many-core CPUs. Demo'ed at several public events like
  Intel IDF, Siggraph, \dots}{}

\cvproject{2005--2007}{Utah Ray Tracing Center of Excellence} {The State of Utah
Ray Tracing Center of Excellence is fundet by the State of Utah, with
the explicit aim of commercializing the ray tracing IP that is beeing
developed at the University of Utah.  While established prior to my
arrival in Utah, this explicitly includes the recently developed
techniques for fast packet/frustum traversal and for handling dynamic
scenes (DynRT).}{}

\cvproject{2006--2007}{Dynamic Ray Tracing for visualizing large
  and time-varying data sets} { Largely as part of the CSafe project
  (CSafe - Center for the Simulation of Advanced Fires and
  Explosions), this project concentrated on interactively visualizing
  large and time-varying data sets based on the DynRT codebase. During
  the course of this (ongoing) project we developed methods for
  visualizing particle data sets, as well as large and time-varying
  volume data sets in both structured and unstructured form.}{}

\cvproject{2005--2007}{DynRT: Ray Tracing Dynamic Scenes} {
Development of a ray tracing architecture that is particularly
optimized for SMP desktop PCs, and which allows for
ray tracing dynamically animated scenes via the development of new,
unconventional acceleration data structures. The project developed
several new approaches to ray tracing animated scenes, while at the
same time staying at least competitive with the fastest known ray
tracers even for static data sets. It led to several papers at
prestigeous places like SIGGRAPH, TOG, Eurographics, etc. Since the
developed methods are also competitive. The developed codebase forms
the core for most of the recent ray tracing based rendering and
visualization projects I was later on involved in.}{}

\cvproject{SaarCOR\\  2001--2003\\~\\  Utah HWRT\\2006--2007}
{Design of special-purpose Ray Tracing Hardware}
%
{I have been actively involved in two different hardware ray tracing
  projects, one at Saarbr\"ucken, and on at Utah. At Saarbr\"ucken,
  the ``SaarCOR'' project started as a spin-off project off the OpenRT
  software ray tracing project. It was started by myself, J\"org
  Schmittler, and Philipp Slusallek, and developed a hardware
  implementation of the OpenRT software core.
%
Unrelated to the SaarCOR project, the University of Utah has also
started a hardware ray tracing project that was originally proposed as
a collaborative project between Peter Shirley, Steven Parker, Erik
Brunvand, and Al Davis prior to me being at Utah. The project finally
got fundet and started in early 2006, after which I joined it.}{}

\cvproject{2003--2006}{inTrace -- Ray Tracing Solutions for
  Industrial Applications} { Based on the OpenRT realtime ray tracing
  engine, we developed several ray tracing based solutions to
  practical industrial problems, including the interactive
  visualization of massively complex engineering models (VW, Boeing),
  as well as interactive photorealistic rendering of important
  lighting effects like shadows, reflections, glass, etc, that
  current, non-ray tracing based solutions cannot accurately simulate.
  This project led to the foundation of inTrace realtime ray tracing
  technologies GmbH, which commercializes this technology. The
  technology is already being used by a large number of high-end
  customers, including VW, Audi, DaimlerChrysler, EADS, BMW,
  etc.  }{}

\cvproject{2002--2005}{Fast and Interactive Global Illumination} {
With the availability of a fast ray tracing architecuture (OpenRT, see
below), it was a natural step to also use this architecture for
simulating global illumination. This project led to a variety of
interactive global illumination systems, including Instant Radiosity,
a method to interactively compute global illumination in highly
complex scenes, interactive photon mapping, and severeal precomputed
global illumination techniques. Due to Utah having a stronger focus on
visualization problems, I mostly stopped working on this project after
moving to Utah, but recent trends in rendering seem to make a revival
of this topic imminent; first steps towards interactive distribution
ray tracing are already being done.} {}

\cvproject{2000--2006}{OpenRT}
{OpenRT is a highly versatile software system that enables realtime ray
  tracing applications on PCs and PC cluster.
%
The OpenRT project was envisioned, designed, and realized in the
  course of my PhD thesis. Since its foundation in 2000, the project
  was continuously extended; while originally based on a highly
  optimized implementation of the ray tracing algorithm only, it has
  since then grown to support a whole set of additional services:
  visualization of highly complex datasets, a out-of-core rendering
  module, direct visualization of spline surfaces, parallel and
  distributed rendering, support for volume and isosurface data sets,
  point-based rendering, interactive visualization of indirect
  illumination and caustics, etc. The OpenRT project has led to
  several spin-off projects (the MassiveRT project, SaarCOR, Instant
  Global Illumination, etc), provided the basis for founding the
  \emph{in}Trace Realtime Ray Tracing GmbH, led to more than 30
  OpenRT-based papers, and 5 different PhD theses.}{}

\cvproject{1997--1999}{McRender}{McRender is a Monte Carlo-based global
  illumination rendering package designed and developed by Alexander
  Keller at the University of Kaiserslautern. I was first involved in
  the project as a research assistant, in the course of which I added
  several new ray tracing data structures and traversal methods, as
  well as helped in developing various hierarchical global
  illumination schemes. During my masters thesis, I also extended it
  to handle Photon Mapping, including several new importance sampling
  scehemes that were published later on.}{}


\closesection
\medskip

%\pagebreak
% \firstnamestyle{\Huge{Industrial Experience}}
% \vspace*{6mm}

% \section{}

% %\cventry{2000--2004}{
% %Dr.-ing.%
% %%PhD in Computer Science
% %}{Saarland University,
% %  Saarbr\"ucken, Germany}
% %{}{}{Awarded Degree: Dr.-ing. mit Auszeichnung (PhD in Computer Science, with distinction)%}

% \cventry{Nov 2007--current}{Intel Corp}{Corporate Technology Group, Advanced Graphics Architectures (CTG/MTL/ARL); Visual Applications Research Lab (VAR)}{}{}{Senior Research Scientist}

% \cventry{May--Aug 2007}{Intel Corp}{Santa Clara, CA,
%   USA}{}{}
% {as a Visiting Professor in the Advanced Graphics Architectures Lab.}

% \cventry{2003--2006}{\emph{in}Trace Realtime Ray Tracing
%   GmbH}{Saarbr\"ucken, Germany}{}{}
% {\ende{Co-founder and Vice-CEO.}{Mitgr\"under und zweiter
%     Gesch\"aftsf\"uhrer}}

% \cventry{May--Aug 2002}{Intel Corp}{Santa Clara, CA, USA}{}{}
% {in the Microprocessor Research Labs / Advanced Graphics
%   Architectures.}

% \cventry{Feb--Mar 1997}
% {Daimler-Benz Aerospace (DASA)}{Bremen, Germany}{}{}
% {Internship, Division of Space\&Infrastructure, DASA, Bremen, Germany.
% }
% \closesection



\bigskip
\bigskip
\bigskip


\ifx\empty
\pagebreak
\firstnamestyle{\Huge{Co-Supervised Students}}
\vspace*{6mm}

I only became officially eligible for serving on PhD commitees after
my visiting professor's appointment was changed to a research
professorship in the summer of 2006. Before that date, I technically
co-advised students, but did not officially serve on their respective
committees. 
\\[1ex]

\section{On PhD Committees for}

\cventry{2006--current}{Dave Edwards}{School of Computing, University
  of Utah}{main adviser P.\
  Shirley}{}{Sampling for ray based rendering.}

\cventry
{2006--current}
{Aaron Knoll}
{SCI Institute, University of Utah}
{main adviser Charles D.\ Hansen}{}
{Interactive Visualization of large and time-varying volume data.}
\cventry{2006--current}{Andrew Kensler}{SCI Institute, University of
  Utah}{main adviser S.\ Parker}{}{Fast traversal and Intersection
  methods.}
\cventry
{2005--current}
{Thiago Ize}
{SCI Institute, University of Utah}
{main adviser C.\ Hansen}{}
{Algorithms and Data Structures for Dynamic Geometry and Parallel Ray Tracing Systems.}
\cventry
{2005--current}
{Vincent Pegoraro}
{SCI Institute, University of Utah}
{main adviser S.\ Parker}{}
{Efficient Monte-Carlo Methods for Global Illumination and Participating Media.}

\closesection

\section{Co-Supervised PhD Students {\small (before having an academic appointment)}}
\cventry
{2005--current}
{Solomon Boulos}
{School of Computing, University of Utah}
{princ.\ adv.\ P.\ Shirley}{}
{Ray tracing dynamic scenes, interactive ray
  tracing of advanced illumination effects.}
\cventry
{2001--2005}
{Johannes G\"unther}
{Saarland University}
{princ.\ adv.\ Ph.\ Slusallek}{}
{{Realtime} {Ray} {Tracing} on current {CPU} {Architectures}
  (successfully defended in 2006)}
\cventry
{2001--2005}
{J\"org Schmittler}
{Saarland University}
{princ.\ adv.\ Ph.\ Slusallek}{}
{{SaarCOR} - {A} {Hardware}-{Architecture} for {Realtime} {Ray} {Tracing} (successfully
  defended in 2005).}
\cventry
{2004--2005}
{Johannes G\"unther}
{MPI Informatik}
{princ.\ adv.\ H-P.\ Seidel}{}
{Ray tracing dynamic scenes, interactive global illumination.}
\cventry
{2004--2005}
{Heiko Friedrich}
{Saarland University}
{princ.\ adv.\ Ph.\ Slusallek}{}
{Ray tracing for future games and volume visualization.}


\closesection
\section{Co-Supervised Masters Students {\small (before having an academic appointment)}}
\cventry
{2004--2005}
{Stefan Sch\"uffler}
{Saarland University}
{princ.\ adv.\ Ph.\ Slusallek}{}
{OSRT - An Open Source implementation of the OpenRT API.}
\cventry
{2000--2001}
{Carsten Benthin}
{Saarland University}
{princ.\ adv.\ Ph.\ Slusallek}{}
{Interactive {Distributed} {Ray} {Tracing} of {Highly} {Complex} {Models}.}
\cventry
{2002--2003}
{Johannes G\"unther}
{Saarland University}
{princ.\ adv.\ Ph.\ Slusallek}{}
{{Realtime} {Caustics} {using} {Distributed} {Photon} {Mapping}.}
\cventry
{2002--2004}
{Heiko Friedrich}
{Saarland University}
{princ.\ adv.\ Ph.\ Slusallek}{}
{Interactive Iso-Surface Ray Tracing using Implicit KD-Trees.}
\cventry
{2001--2002}
{Markus Wagner}
{Saarland University}
{princ.\ adv.\ Ph.\ Slusallek}{}
{Development of a {Ray-Tracing-Based} {VRML} {Browser} and {Editor}.}



\closesection
\fi

\end{document}


%% end of file `jdoe_casual.tex'.
